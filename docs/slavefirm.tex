\section{Slave firmware}
\boxy{You probably don't need to read this section unless you're maintaining
the code for the slave controllers.}
The slave firmware processes incoming \isqc{} messages,
reads encoders and potentiometers, and
handles control of the motors. It's the most complex part
of the code.


\subsection{I2C slave code (slave\_i2c.h)}
This code implements the communications as an abstract class and 
a namespace.

\subsubsection{I2CSlave namespace}
This sets up \isqc{} and handles communications from the master using listener classes (actual communications are
handled by the Wire library and interrupt callbacks.)
The methods are:
\begin{itemize}
\item \textbf{init} initialises comms with a given device address, which must be
unique for each device. It also takes a pointer to the register table, which is in the same format as that passed
to the master firmware's \emph{I2CSlave} class constructor, and is stored in the \textbf{regs.ino} file
in the common directory.
\item \textbf{setRegister} changes the value of a register, so that it can then be read by the master.
\item \textbf{addListener} adds an event listener (see below) to the system, which will be informed on register writes.
\item \textbf{poll} should be called periodically, and checks whether any registers have been changed since it
was last called. If so, it will call the listener with the register index and new value.
\end{itemize}

\subsubsection{I2CSlaveListener}
This is an abstract class, implementations of which can be added to the list of listeners (see above.)
When a register has changed, the \emph{changeEvent()} method is called with the register index and new value ---
but only when \emph{poll()} is called.

There is also a \emph{changeEventImmediate()} callback, which will be called
from inside the interrupt as soon as the new data arrives. Because it requires
a little more processing, this mechanism is disabled by default.

Therefore, when a register is changed, the hardware will only take action when the listener's \emph{changeEvent()} is called
at the next poll. \textbf{This may lead to a very slight delay.} 

\subsubsection{I2CLoop() in i2c.ino}
This is called repeatedly during the run, as fast as possible. It does the following:
\begin{itemize}
\item Calls \emph{poll()} so that received messages will cause registers to change. These changes will be handled by the
\emph{MyI2CListener} object --- for example, writing a required speed to the REG\_SPEED register for a speed motor
will cause \emph{setRequiredSpeed()} to be called with that value on that motor.
\item Sets monitoring registers with values from the motors.
\end{itemize}



\subsubsection{Low-level I2C code}
Callbacks are registered with the Wire library, which are called from the I2C interrupt inside the library. They work thus:
\begin{itemize}
\item \textbf{receiveEvent} is used when data is written to the slave. The current register is set to
the first byte of the data, and if there are any more bytes, these are written to the register using the
methods given above.
This causes the register's changed bit to be set, so that \emph{poll()} will call a listener when next called.
\item \textbf{requestEvent} is used when a request is made for data. It assumes that the \emph{receiveEvent()} has been called, and has set the current register number, and writes the value
of the register to the bus. If the register doesn't exist a dummy value (\texttt{0xcd)} is written.
\end{itemize}
It follows therefore that on both master and slave \emph{reading is more expensive than writing,} because two things
happen: a message is received which sets the register number, and then data is requested.

\subsection{Motors (motor.h)}
A \emph{MotorController} class is used as the basic core of motor functionality. It has:
\begin{itemize}
\item the enable, positive and negative pin assignments;
\item methods to set both pins high or low for braking;
\item methods to set the speed;
\item methods for current monitoring including hysteresis using a geometric decay;
\item a pointer to a chain of \emph{MotorListeners} for handling overcurrent events, with
a programmable overcurrent threshold.
\end{itemize}

\subsection{PIDController (pid.h)}
This class wraps the \emph{MotorController} class with PID calculations,
used by both speed and position motors. The PID results
are used slightly differently by the two subclasses.
It provides:
\begin{itemize}
\item methods for setting the desired value (speed or position);
\item methods for setting the gains;
\item monitoring methods (e.g. \emph{getError()};
\item an \emph{update()} method which must be called every tick;
\end{itemize}


\subsection{Speed motors (speedmotor.h)}
\emph{SpeedMotorController} is an extension of \emph{PIDController} which
uses a PID controller for speed, using a quadrature encoder as the sensor. As such,
it provides:
\begin{itemize}
\item a \emph{tickEncoder()} method which must be called whenever the encoder's A channel gets a rising edge;
\item a method to be called when ADC on the current monitors is completed, to update that value.
\end{itemize}

\subsubsection{Encoders (quadencoder.h) and the encoder interrupt}
The \emph{QuadEncoder} class is normally used only by \emph{SpeedMotorController}. It assumes that the encoder is wired
to port D, pins 0 and 1 for channels A and B respectively (usually these are the RX and TX pins, or digital pins 0 and 1.)
This means that each controller slave can only handle a single encoder motor.

An interrupt is required to use the encoder --- on port D pin 0 rising edge, the \emph{tickEncoder()} method
must be called on the \emph{SpeedMotorController} instance. This is set up thus in \textbf{sketch.ino}:
\begin{v}
void setupEncoderISR(){
    cli();
    PCMSK2 |= 1<<PCINT16; // PD-0.
    PCICR |= 1<<PCIE2; // turn on ints for port D
    sei();
}
ISR(PCINT2_vect){
    if(PIND & 1) // rising edge on pin 0
        driveMotor.tickEncoder();
}
\end{v}

\subsection{Position motors (posmotor.h) }
\emph{PositionMotorController} is an extension of \emph{PIDController} which uses the PID controller
for positional control, using a potentiometer.

\subsubsection{Calibration}
Each positional motor's potentiometer is connected to an ADC which produces a value in the range 0-1023. This value
is mapped onto a user-specified range, given by the calibration registers which are initialised to the range
-512 to 512.

\subsection{ADC (adc.h/adc.ino)}
The \emph{ADCReader} class simplifies the process of running
a series of free-running ADC reads. Only one read can be running at a time,
so the ADC reader cycles through multiple reads which the developer can register
with it. When a read
completes, it calls a method in a \emph{ADCListener} class
giving the value, and starts the next read running. Methods are:
\begin{itemize}
\item \textbf{init()} : initialise (for which interrupts must be disabled)
\item \textbf{poll()} which must be called regularly
\item \textbf{addRead(channel,type,listener)} : register a new read, giving
the ADC channel (i.e. the pin number), a type (which is an integer whose meaning is determined
by the developer) and a pointer to the listener object. The listener is some implementation of the
\textbf{ADCListener} interface, which will have a \textbf{onADC(type, value)} call.
\end{itemize}
ADC listeners are registered for all motors for current monitoring, the steer and lift motors for
position monitoring, and the chassis system for inclinometer readings.

\subsection{State}
The \texttt{State} class handles exception states on both the slave
and master, and is declared in \texttt{common.h} because it's
used by both (with slight modifications.) Exceptions, when they
occur, cause both a change in the state object and any exception
listener registered with it to be notified.

Exceptions are discussed further in section~\ref{safta}.

\subsection{Main code (sketch.ino)}
The main code consists of:
\begin{itemize}
\item A listener which sets LEDs when an exception occurs;
\item Two motor instances (speed and position for drive/steer controllers, or two position motors for lift/lift 
controllers;)
\item An array of pointers to those motor instances, for convenience;
\item various routines for handling PWM rate changing, and ADC work;
\item Setup code;
\item The main loop.
\end{itemize}
\subsubsection{Setup}
\begin{itemize}
\item Disable pullups for I2C and encoder
\item Enable watchdog timer
\item Set pin modes
\item Read the I2C address from EEProm 
\item Flash some lights
\item Change the PWM frequency for the motors
\item Setup the two motors with their pin assignments
\item Register the ADC listeners
\item Initialise ADC
\item Set up the interrupt for the encoder if this is a drive/steer slave
\item Initialise I2C comms
\end{itemize}
\subsubsection{Main loop}
\begin{itemize}
\item Poll the I2C system, setting monitoring registers to the values and calling update on the motors.
\item Poll the ADC system, updating data on completion of an ADC and moving on to reading another value.
\item Handle debugging LEDs
\end{itemize}

