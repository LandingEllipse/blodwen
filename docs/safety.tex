\section{Safety}
\label{safta}
Safety is achieved through two mechanisms:
\begin{itemize}
\item \textbf{Constraints} can be set on certain quantities
which are checked at all possible levels;
\item \textbf{Exceptions} are raised when certain conditions are met or
constraints are violated. It may be possible to stop certain contraints
from throwing exceptions.
\end{itemize}

\subsection{Input constraints}
The simplest constraints are the limits placed on input values, typically the
required positions and speeds. These are already adequately handled by the
register mapping system --- if an attempt is made to set a register to a value
which is outside its mapped range, a \emph{SlaveException} is thrown by
the PC code.

\subsubsection{Lift constraints}
Lift constraints are more complex, as they involve detecting combinations
of required inputs which would cause a clash of legs. This is done at the PC
API level, in the \texttt{isAdjacencyViolated()} method of \emph{LiftMotor.} 

This looks at adjacent pairs of legs when \texttt{setRequired()} is called,
and if one leg is requested negative
while the leg on its negative side is requested positive (or \emph{vice versa})
a \emph{ConstraintException} is thrown.

\subsection{Rover exceptions}
Another set of constraints operate at the slave level to detect stalls,
encoder malfunctions and other problems. They work by examining the speed at
which the motor is moving, the control signal being sent to the motor, and the
current flowing. The errors produced, if any, in all combinations of these
values are are shown in figure~\ref{errorstatestab}.

\begin{center}
\fbox{\parbox{4in}{
Currently this code is \textbf{disabled} because it leads to too many
false positives.
This has been done by setting the \texttt{STALLCHECK} value to 255 --- the
(unsigned 8 bit) ``stall state counter,'' which is incremented each tick
the system is in such a state, must exceed this value before a
stall/fault is detected.
}}
\end{center}


\begin{figure}[ht]
\center
\begin{tabular}{|l|l|l|l|}\hline
Speed & Control value & Current & Result \\ \hline
Stop & Low & Zero & \\
Stop & Low & Low & \\
Stop & Low & Nominal/High & SHORT\\
Stop & High & Zero &  FAULT (drive)\\
Stop & High & Low &  FAULT (encoder)\\
Stop & High & Nominal & STALL \\
Run & Low & Zero & ? \\
Run & Low & Low & ? \\
Run & Low & Nominal & ? \\
Run & High & Zero & ? \\
Run & High & Low &  \\
Run & High & Nominal &  \\ 
- & High & High & OVERCURRENT\\ \hline
\end{tabular}
\caption{Error table}
\label{errorstatestab}
\end{figure}
\begin{itemize}
\item \textbf{Short} --- there is a short in the drive wires, so a high
current is being passed without the motor turning;
\item \textbf{Fault} --- there is either a break in the drive wires;
(in which case the current will be zero) or a fault in the encoder or
encoder wires (in which case the current will be non-zero);
\item \textbf{Stall} --- the motor is unable to move;
\item \textbf{Overcurrent} --- too high a current through the motor;
\item \textbf{?} --- probably inertial effects, ignore.
\end{itemize}
When such a condition occurs, the slave goes into the exception state
and informs the master. The master then transmits a message to all the
slaves, so they also go into an exception state. Exception data includes:
\begin{itemize}
\item The motor ID on the controller (0 or 1)
\item The type of the exception
\end{itemize}
After an exception, therefore, the slaves will all be in an exception state.
Most will be showing the REMOTE exception, but one will show the
original problem and the motor for which it occurred.

Exception states and transitions are shown in Figure~\ref{exceptionfig}. Dotted
lines indicate causation, so a dotted line from a transitions to another
transition indicates that the first transition causes the second.
\begin{figure}[ht]
\center
\includegraphics[width=5in]{exceptions.pdf}
\caption{Exception states}
\label{exceptionfig}
\end{figure}


\subsubsection{Exception transitions}
Firstly, both the slaves and master \textbf{start in the exception state}. This
is because, in the case of a communications error, the \isqc{} routines
will stall, causing the processor to reboot when the watchdog times out.

It is the responsibility of the PC library to reset the all slaves, and then
reset the master, when the system starts up. If an exception occurs
later, the master and slaves will once again be in the exception state
and the PC code will again have to reset them.

If a ``normal'' exception (i.e. not a communications error) occurs
on a slave, the slave goes into the exception state and reports the error
to the master, which also goes into the exception state and tells all other
slaves to do the same. The exception registers on the master will hold
information about the original slave's exception.

In addition, the failure
of a slave to respond to a message will cause it to reboot, which will
cause the master to reboot (this is the only failure mode possible for \isqc{}) which will send 
RESET messages to all slaves. The result is all devices in exception.

\subsubsection{Exception actions}
On entry to the exception state, the master signals all the slaves to
also go into exception. The slaves, in an exception state, do the following:
\begin{itemize}
\item Set both negative and positive pins of the motors to LOW
\item Set the duty cycle on the enable pins to zero
\end{itemize}
This causes the motors to perform an unbraked stop.

In the exception state, register values can be changed but will have no effect
until the exception is reset, because the control loop does not run in
exception\footnote{In an earlier version, register changes were ignored. This
led to the problem of being unable to recover from an error resulting from a
bad register setting (such as overcurrent threshold to zero.)}. It's important
to note that the \textbf{boot exception will have set all values to their
defaults.} This is not the case for other exceptions.


\subsubsection{State flags}
The exception substates for the master are:
\begin{itemize}
\item Boot state (could be caused by \isqc{} comms failure)
\item Problem reported by slave (ID and problem specified)
\end{itemize}
The exception substates for the slaves are:
\begin{itemize}
\item Boot state (powerup, or signalled by master using RESET command)
\item Local problem as shown in figure~\ref{errorstatestab}
\item Problem with another slave, reported by master
\item Halt state (used when a comms error occurs, forces the master to reboot)
\end{itemize}

