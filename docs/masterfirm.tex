\section{Organisation of firmware source code}
\boxy{You probably don't need to read this section unless you're maintaining
the code for the master controller.}

The source code is kept in the \texttt{firmware} directory:
\begin{itemize}
\item \textbf{common} holds useful definitions for both slave and master, such as register 
structures and assignments.
\item \textbf{master} holds code for the master (Arduino) side;
\item \textbf{slave} holds code used by the slave (motor driver) side.
\end{itemize}
\subsection{Common register table}
The \emph{common} directory contains files describing the registers used
in the system:
\begin{itemize}
\item \textbf{regDefinitions} is a file with a special syntax from which
\texttt{regsauto.h}, \texttt{regsauto.cpp} and \texttt{regsauto.ino} are
built by the \texttt{build} script --- see below for details.
\item \textbf{regs.h} defines the \emph{Register} structure, which
encapsulates a one or two byte value which may or may not be mapped onto
a arbitrary range of floating point values (see below).
\item \textbf{regsauto.h}, \textbf{regsauto.cpp} and \textbf{regsauto.ino}
are generated automatically by the \emph{build} script. They contain
register definitions --- the include file contains the register names and
numbers, while the identical \texttt{cpp} and \texttt{ino} files contain
the \emph{Register} structures describing each register: number of bits,
writability, float range (or values indicating ``unmapped'').
These files are compiled into master, slave and PC code using symbolic links
from their respective source directories.
\end{itemize}
\subsection{Register definition file}
The registers are defined in the register definition file
\texttt{regDefinitions} which can be found in the \texttt{firmware/common} 
directory. It consists of a number of blocks, each of which
specifies a different register table (the two types of slave
controller have different register sets.) Named blocks are
preceded by an unnamed block, which specifies a block
of registers common to all tables.

A Haskell script, \texttt{regparse.hs}, is used to generate
the \texttt{.h} and \texttt{.cpp/ino} files automatically,
which is itself invoked by running \texttt{build.} 

\subsubsection{Register value mapping}
There are two types of register: \textbf{unmapped}, which are straightforward
8 or 16 bit values (depending on the size of the register); and \textbf{mapped},
where a floating point range is mapped onto the underlying range.

These mappings are defined in the register definition file. For example:

\begin{v}
    RESET          writable  1 unmapped      "reset bits"
\end{v}
defines a register called RESET, which is a writable 1-byte unmapped value;
whereas
\begin{v}
    DRIVE_REQSPEED writable  2 range -4000 4000  "required speed"
\end{v}
defines a register in which the floating point range [-4000,4000] is mapped onto
the fixed point range [0,65535].
Mapping and unmapping are done using the \texttt{map()} and \texttt{unmap()} methods
in Register; the PC and firmware deal with this automatically as required (see, for example,
the \texttt{i2c\_interface.ino} code in the slave, which catches register changes as they happen,
unmaps them, and uses them to set values in the control system.)

If map or unmap are called on an unmapped register, the value is unchanged (apart from its type.)

\clearpage
\section{Master firmware and PC-Master protocol}
\label{protoc}
The master firmware takes commands from the USB serial port and communicates
with the slaves over \isqc{}.
\subsection{PC-Master protocol}
This protocol is a binary serial protocol. Messages from
the PC to the master have the following form:
\begin{itemize}
\item 1 byte: number of bytes in packet (including this one)
\item 1 byte: command number and slave ID. The slave ID occupies
the top 4 bits, the command number the bottom 4. If the slave
ID is zero, the registers to be accessed are local registers
on the master. The commands are:
\begin{itemize}
\item 2 : write command
\item 3 : read command
\item 4 : set read set command
\end{itemize}
\item Remaining bytes: payload, see below
\end{itemize}
The response depends on the command, as described in the following
subsections which go into the details of  the each command.
\subsubsection{Write command, code 2}
Payload:
\begin{itemize}
\item 1 byte: number of writes. \textbf{For each write:}
\begin{itemize}
\item 1 byte: register number
\item 1-2 bytes (depending on size of register,
obtained from register table) : value of register, LSB first
\end{itemize}
\end{itemize}
Response: 1 byte, value zero.
\subsubsection{Read command, code 3}
Payload is a single byte, the index of the read set to read. Response is, \textbf{for each register in
the given read set:}
\begin{itemize}
\item 1-2 bytes (depending on size of register,
obtained from register table) : value of register, LSB first
\end{itemize}
\subsubsection{Set read set command, code 4}
The first byte in the payload is index of the read set to change. 
Each subsequent byte in the payload is the index of a register
which should be in the read set. The next read command for that read set
will result in the value of these registers being sent,
in the same order, to the PC. \textbf{Read sets are shared across all devices}
so the slave number is irrelevant.

Response is 1 byte: the number of registers in the new read
set.

\clearpage
\subsection{Examples of PC-Master protocol commands}
For the following examples, registers 0 and 1 are 1 byte
registers, while 2 and 3 are 2 byte registers.

\subsubsection{Example 1: write}
PC to master: \verb+0d 32 04 00 ff 01 ff 02 01 00 03 02 00+ \\
Response: \verb+00+ 
\begin{itemize}
\item \verb+0d 32+:  13 bytes, command 2 (write) for slave 3
\item \verb+04+: contains 4 writes
\item \verb+00 ff+ write \verb+ff+ to register 0
\item \verb+01 ff+ write \verb+ff+ to register 1
\item \verb+02 01 00 + write \verb+0001+ to register 2
\item \verb+03 02 00 + write \verb+0002+ to register 3
\end{itemize}
Response is just zero to acknowledge.

\subsubsection{Example 2: set read set}
PC to master: \verb+06 04 01 00 01 02 03+ \\
Response: \verb+04+ 
\begin{itemize}
\item \verb+06 04+: 6 bytes, command 4 (set read set) for slave 0 (this value is ignored.)
\item \verb+01+: change read set 1
\item \verb+00 01 02 03+: registers 0, 1, 2 and 3 should be the new read set 1.
\end{itemize}
The response is \verb+04+, the number of registers in the read set.

\subsubsection{Example 3: read}
PC to master: \verb+03 12 01+ \\
Response: \verb+ff ff 01 00 02 00+ 
\begin{itemize}
\item \verb+03 12+: 3 bytes, command 2 (read) for slave 1
\item \verb+01+: read set 1
\end{itemize}
The response is the values of the registers:
\begin{itemize}
\item \verb+ff+ : register 0 contains \verb+ff+ 
\item \verb+ff+ : register 1 contains \verb+ff+ 
\item \verb+01 00+ : register 2 contains \verb+0001+ 
\item \verb+02 00+ : register 3 contains \verb+0002+ 
\end{itemize}

\subsection{Firmware registers}
\label{firmrec}
%
%These are not to be confused with the top-level fa\c{c}ade registers, which act as a high-level
%network capable API. These firmware registers are only used in communications between the PC
%and the \isqc{} units, mediated by the master.

%The reason for these two sets of registers is that at a low level, each wheel is controlled
%by more than one slave controller: the drive/steer slave for that wheel, and the lift/lift slave
%for the wheel pair. This level of complexity --- knowing which controller to address
%to affect which motor on which wheel --- is handled at a higher level by the on-board PC.
%In that system, both a set of C++ classes and a set of registers for network control
%are used which translate into read/write commands for the firmware registers.

In the following tables, the columns have the following meanings:
\begin{itemize}
\item \textbf{ID} : the ID number of the register
\item \textbf{Symbol} : the C++ \verb+#define+ symbol given to that ID
\item \textbf{n} : the number of bytes for this register
\item \textbf{W} : whether the register is writable or not
\item \textbf{Mapping} : if the register is mapped onto a floating point range, the interval is specified. ``Unmapped'' indicates an integer or bitfield register.
\item \textbf{Description} : a brief description
\end{itemize}



\subsubsection{Common block}
These registers are common to all devices \emph{except} the master device, which is the local master
controller and has the special ID 0.

\begin{tabular}{|p{0.2in}|p{2.7in}|p{0.1in}|p{0.1in}|p{1in}|p{1.5in}|}\hline
\textbf{ID} & \textbf{Symbol} & \textbf{n} & \textbf{W} & \textbf{Mapping} & \textbf{Description}  \\ \hline 
0 & REG\_RESET & 1 & Y & unmapped & reset bits - beware race conditions\\ \hline
1 & REG\_TIMER & 2 &  & unmapped & millis since start\\ \hline
2 & REG\_INTERVALI2C & 2 &  & unmapped & interval between I2C ticks\\ \hline
3 & REG\_STATUS & 2 &  & unmapped & see status flags in regs.h\\ \hline
4 & REG\_DEBUGLED & 1 & Y & unmapped & debugging LEDs, turns on for some time\\ \hline
5 & REG\_EXCEPTIONDATA & 2 & Y & unmapped & LSB: type, MSB: id. Write causes REMOTE exception\\ \hline
6 & REG\_DISABLEDEXCEPTIONS & 2 & Y & unmapped & bitfield of disabled exceptions\\ \hline
7 & REG\_PING & 1 & Y & unmapped & debugging\\ \hline
8 & REG\_DEBUG & 2 & Y & unmapped & debugging\\ \hline
\end{tabular}

\clearpage
\subsubsection{Drive/steer registers}
These registers appear on drive/steer controllers, and control a drive and steer motor.
There may also be a chassis potentiometer, although it is only present on one of the two D/S slaves
in each wheel pair.

\begin{tabular}{|p{0.2in}|p{2.7in}|p{0.1in}|p{0.1in}|p{1in}|p{1.5in}|}\hline
\textbf{ID} & \textbf{Symbol} & \textbf{n} & \textbf{W} & \textbf{Mapping} & \textbf{Description}  \\ \hline 
9 & REGDS\_DRIVE\_REQSPEED & 2 & Y & [-4000.0,4000.0] & required speed\\ \hline
10 & REGDS\_DRIVE\_PGAIN & 2 & Y & [0.0,10.0] & P-gain\\ \hline
11 & REGDS\_DRIVE\_IGAIN & 2 & Y & [0.0,10.0] & I-gain\\ \hline
12 & REGDS\_DRIVE\_DGAIN & 2 & Y & [-10.0,10.0] & D-gain\\ \hline
13 & REGDS\_DRIVE\_INTEGRALCAP & 2 & Y & [0.0,1000.0] & integral error cap\\ \hline
14 & REGDS\_DRIVE\_INTEGRALDECAY & 2 & Y & [0.0,1.0] & integral decay\\ \hline
15 & REGDS\_DRIVE\_OVERCURRENTTHRESH & 2 & Y & [0.0,1000.0] & overcurrent threshold\\ \hline
16 & REGDS\_DRIVE\_ACTUALSPEED & 2 &  & [-4000.0,4000.0] & actual speed from encoder\\ \hline
17 & REGDS\_DRIVE\_ERROR & 2 &  & [-1000.0,1000.0] & required minus actual speed\\ \hline
18 & REGDS\_DRIVE\_ERRORINTEGRAL & 2 &  & [-1000.0,1000.0] & error integral magnitude\\ \hline
19 & REGDS\_DRIVE\_ERRORDERIV & 2 &  & [-200.0,200.0] & error derivative\\ \hline
20 & REGDS\_DRIVE\_CONTROL & 2 &  & [-255.0,255.0] & value being sent to motor\\ \hline
21 & REGDS\_DRIVE\_INTERVALCTRL & 2 &  & [0.0,1000.0] & time between control runs (ms)\\ \hline
22 & REGDS\_DRIVE\_CURRENT & 2 &  & unmapped & raw current reading\\ \hline
23 & REGDS\_DRIVE\_ODO & 2 &  & unmapped & encoder ticks\\ \hline
24 & REGDS\_DRIVE\_STALLCHECK & 1 & Y & [0.0,255.0] & stall check control signal level\\ \hline
25 & REGDS\_DRIVE\_DEADZONE & 1 & Y & [0.0,50.0] & if below this value, error is set to zero\\ \hline
26 & REGDS\_STEER\_REQPOS & 2 & Y & [-200.0,200.0] & required position\\ \hline
27 & REGDS\_STEER\_PGAIN & 2 & Y & [0.0,100.0] & P-gain\\ \hline
28 & REGDS\_STEER\_IGAIN & 2 & Y & [0.0,10.0] & I-gain\\ \hline
29 & REGDS\_STEER\_DGAIN & 2 & Y & [-10.0,10.0] & D-gain\\ \hline
30 & REGDS\_STEER\_INTEGRALCAP & 2 & Y & [0.0,1000.0] & integral error cap\\ \hline
31 & REGDS\_STEER\_INTEGRALDECAY & 2 & Y & [0.0,1.0] & integral decay\\ \hline
32 & REGDS\_STEER\_OVERCURRENTTHRESH & 2 & Y & [0.0,1000.0] & overcurrent threshold\\ \hline
33 & REGDS\_STEER\_ACTUALPOS & 2 &  & [-200.0,200.0] & actual position from pot\\ \hline
34 & REGDS\_STEER\_ERROR & 2 &  & [-200.0,200.0] & required minus actual position\\ \hline
35 & REGDS\_STEER\_ERRORINTEGRAL & 2 &  & [-1000.0,1000.0] & error integral magnitude\\ \hline
36 & REGDS\_STEER\_ERRORDERIV & 2 &  & [-200.0,200.0] & error derivative\\ \hline
37 & REGDS\_STEER\_CONTROL & 2 &  & [-255.0,255.0] & value being sent to motor\\ \hline
38 & REGDS\_STEER\_INTERVALCTRL & 2 &  & [0.0,1000.0] & time between control runs (ms)\\ \hline
39 & REGDS\_STEER\_CURRENT & 2 &  & unmapped & raw current reading\\ \hline
40 & REGDS\_STEER\_STALLCHECK & 1 & Y & [0.0,255.0] & stall check control signal level\\ \hline
41 & REGDS\_STEER\_DEADZONE & 1 & Y & [0.0,50.0] & if below this value, error is set to zero\\ \hline
42 & REGDS\_STEER\_CALIBMIN & 1 & Y & [-120.0,120.0] & minimum angle, mapped onto pot value 0\\ \hline
43 & REGDS\_STEER\_CALIBMAX & 1 & Y & [-120.0,120.0] & maximum angle, mapped onto pot value 1024\\ \hline
44 & REGDS\_CHASSIS & 2 &  & [0.0,1024.0] & chassis pot reading\\ \hline
\end{tabular}

\clearpage
\subsubsection{Lift/lift registers}
These registers appear on lift/lift controllers, and control two lift motors.

\begin{tabular}{|p{0.2in}|p{2.7in}|p{0.1in}|p{0.1in}|p{1in}|p{1.5in}|}\hline
\textbf{ID} & \textbf{Symbol} & \textbf{n} & \textbf{W} & \textbf{Mapping} & \textbf{Description}  \\ \hline 
9 & REGLL\_ONE\_REQPOS & 2 & Y & [-200.0,200.0] & required position\\ \hline
10 & REGLL\_ONE\_PGAIN & 2 & Y & [0.0,100.0] & P-gain\\ \hline
11 & REGLL\_ONE\_IGAIN & 2 & Y & [0.0,10.0] & I-gain\\ \hline
12 & REGLL\_ONE\_DGAIN & 2 & Y & [-10.0,10.0] & D-gain\\ \hline
13 & REGLL\_ONE\_INTEGRALCAP & 2 & Y & [0.0,1000.0] & integral error cap\\ \hline
14 & REGLL\_ONE\_INTEGRALDECAY & 2 & Y & [0.0,1.0] & integral decay\\ \hline
15 & REGLL\_ONE\_OVERCURRENTTHRESH & 2 & Y & [0.0,1000.0] & overcurrent threshold\\ \hline
16 & REGLL\_ONE\_ACTUALPOS & 2 &  & [-200.0,200.0] & actual position from pot\\ \hline
17 & REGLL\_ONE\_ERROR & 2 &  & [-200.0,200.0] & required minus actual position\\ \hline
18 & REGLL\_ONE\_ERRORINTEGRAL & 2 &  & [-1000.0,1000.0] & error integral magnitude\\ \hline
19 & REGLL\_ONE\_ERRORDERIV & 2 &  & [-200.0,200.0] & error derivative\\ \hline
20 & REGLL\_ONE\_CONTROL & 2 &  & [-255.0,255.0] & value being sent to motor\\ \hline
21 & REGLL\_ONE\_INTERVALCTRL & 2 &  & [0.0,1000.0] & time between control runs (ms)\\ \hline
22 & REGLL\_ONE\_CURRENT & 2 &  & unmapped & raw current reading\\ \hline
23 & REGLL\_ONE\_CALIBMIN & 1 & Y & [-120.0,120.0] & minimum angle, mapped onto pot value 0\\ \hline
24 & REGLL\_ONE\_CALIBMAX & 1 & Y & [-120.0,120.0] & maximum angle, mapped onto pot value 1024\\ \hline
25 & REGLL\_ONE\_STALLCHECK & 1 & Y & [0.0,255.0] & stall check control signal level\\ \hline
26 & REGLL\_ONE\_DEADZONE & 1 & Y & [0.0,50.0] & if below this value, error is set to zero\\ \hline
27 & REGLL\_TWO\_REQPOS & 2 & Y & [-200.0,200.0] & required position\\ \hline
28 & REGLL\_TWO\_PGAIN & 2 & Y & [0.0,100.0] & P-gain\\ \hline
29 & REGLL\_TWO\_IGAIN & 2 & Y & [0.0,10.0] & I-gain\\ \hline
30 & REGLL\_TWO\_DGAIN & 2 & Y & [-10.0,10.0] & D-gain\\ \hline
31 & REGLL\_TWO\_INTEGRALCAP & 2 & Y & [0.0,1000.0] & integral error cap\\ \hline
32 & REGLL\_TWO\_INTEGRALDECAY & 2 & Y & [0.0,1.0] & integral decay\\ \hline
33 & REGLL\_TWO\_OVERCURRENTTHRESH & 2 & Y & [0.0,1000.0] & overcurrent threshold\\ \hline
34 & REGLL\_TWO\_ACTUALPOS & 2 &  & [-200.0,200.0] & actual position from pot\\ \hline
35 & REGLL\_TWO\_ERROR & 2 &  & [-200.0,200.0] & required minus actual position\\ \hline
36 & REGLL\_TWO\_ERRORINTEGRAL & 2 &  & [-1000.0,1000.0] & error integral magnitude\\ \hline
37 & REGLL\_TWO\_ERRORDERIV & 2 &  & [-200.0,200.0] & error derivative\\ \hline
38 & REGLL\_TWO\_CONTROL & 2 &  & [-255.0,255.0] & value being sent to motor\\ \hline
39 & REGLL\_TWO\_INTERVALCTRL & 2 &  & [0.0,1000.0] & time between control runs (ms)\\ \hline
40 & REGLL\_TWO\_CURRENT & 2 &  & unmapped & raw current reading\\ \hline
41 & REGLL\_TWO\_CALIBMIN & 1 & Y & [-120.0,120.0] & minimum angle, mapped onto pot value 0\\ \hline
42 & REGLL\_TWO\_CALIBMAX & 1 & Y & [-120.0,120.0] & maximum angle, mapped onto pot value 1024\\ \hline
43 & REGLL\_TWO\_STALLCHECK & 1 & Y & [0.0,255.0] & stall check control signal level\\ \hline
44 & REGLL\_TWO\_DEADZONE & 1 & Y & [0.0,50.0] & if below this value, error is set to zero\\ \hline
\end{tabular}

\clearpage
\subsubsection{Master registers}
These registers are local to the master controller --- they are currently used for 
temperature and exception monitoring.

\begin{tabular}{|p{0.2in}|p{2.7in}|p{0.1in}|p{0.1in}|p{1in}|p{1.5in}|}\hline
\textbf{ID} & \textbf{Symbol} & \textbf{n} & \textbf{W} & \textbf{Mapping} & \textbf{Description}  \\ \hline 
0 & REGMASTER\_RESET & 1 & Y & unmapped & set to clear exception state\\ \hline
1 & REGMASTER\_TEMPAMBIENT & 2 &  & [-20.0,100.0] & temperature sensor\\ \hline
2 & REGMASTER\_TEMP1 & 2 &  & [-20.0,100.0] & temperature sensor\\ \hline
3 & REGMASTER\_TEMP2 & 2 &  & [-20.0,100.0] & temperature sensor\\ \hline
4 & REGMASTER\_TEMP3 & 2 &  & [-20.0,100.0] & temperature sensor\\ \hline
5 & REGMASTER\_TEMP4 & 2 &  & [-20.0,100.0] & temperature sensor\\ \hline
6 & REGMASTER\_TEMP5 & 2 &  & [-20.0,100.0] & temperature sensor\\ \hline
7 & REGMASTER\_TEMP6 & 2 &  & [-20.0,100.0] & temperature sensor\\ \hline
8 & REGMASTER\_TEMP7 & 2 &  & [-20.0,100.0] & temperature sensor\\ \hline
9 & REGMASTER\_TEMP8 & 2 &  & [-20.0,100.0] & temperature sensor\\ \hline
10 & REGMASTER\_TEMP9 & 2 &  & [-20.0,100.0] & temperature sensor\\ \hline
11 & REGMASTER\_EXCEPTIONDATA & 2 &  & unmapped & LSB: type, MSB: motor|slave\\ \hline
\end{tabular}



\clearpage
\subsection{Main code (sketch.ino)}
The main code defines and uses a \emph{BinarySerialReader} abstract class,
extended as \emph{MySerialReader,} which processes packets
from the PC in three methods:
\begin{itemize}
\item \textbf{dowrite} handles writing a set of registers for a given slave device
on the \isqc{} bus;
\item \textbf{doread} handles a request to read the registers in the current readset of a device;
\item \textbf{doreadset} handles a request to change the current readset for a device.
\end{itemize}

\subsection{I2CDevice (i2c.h)}
This class encapsulates a link to an \isqc{} slave device, abstracted
as a register interface. Devices may each have their own set of register
definitions, which can include registers of different byte widths and
access controls (i.e. read/write or read only.) Currently, two device types
are supported, as shown in section~\ref{whpairarc}: drive/steer and lift/lift.
While very similar, the registers are slightly different.

The bus address of the device is specified by passing the address into the
constructor.
The registers are also specified in the constructor, by passing a table of
\emph{Register} structures. This table comes from the \emph{regs.ino} file in the common
directory.

\begin{itemize}
\item \textbf{writeRegister} takes the register number
and the new value as an integer --- size conversion is done automatically according
to the size in the register table.
\item \textbf{readRegister} reads data from a register, writing it
to an integer (again, it's auto-resized.) A return value
is provided giving status --- generally a failure means that
a bad register or device number has been passed in.
\end{itemize}
Note that mapping/unmapping are done at a higher level than this --- these two
methods deal with the integer values.
Writing a register does the following:
\begin{itemize}
\item check register exists and is writable
\item begin transmission (i.e. start writing on the bus and send the device number)
\item write the register number and the value
\item end transmission
\end{itemize}
Reading a register does the following:
\begin{itemize}
\item check register exists
\item begin transmission (i.e. start writing on the bus and send the device number)
\item write the register number
\item end transmission
\item request a read from the device
\item wait for data to become available
\item while data is available, read it and add it to the data (which arrives LSB first)
\end{itemize}
It follows, therefore that on both master and slave \emph{reading is more expensive than writing.} 

Note that the constructor for this object will not initialise
the Wire library.

\subsection{Remote control}
Code for the remote is in the following files:
\begin{itemize}
\item \texttt{rc.ino} and \texttt{rc.h} contain the \textbf{RCReceiver} class,
which handles reading servo pulses from the receiver and decoding them.
This is done by using a pin change interrupt handler on the three relevant pins:
we count the number of rising edges since the last update call. There
is also a \textbf{ready()} method which returns true if data was recently
received.
\item \texttt{rcRover.ino} and \texttt{rcRover.h} extend the above class
with methods specific to this application. In particular, the \textbf{update()} method
updates the underlying receiver and sends register writes to the relevant slaves
if there is servo data present.
\end{itemize}
The main loop in \texttt{sketch.ino} (in \textbf{BinarySerialReader::poll()} checks the remote for data, and does
not perform the usual processing --- reading register writes from the PC and
sending them on to the relevant slave --- if there is data. Instead, the 
remote code will start to send register writes.

Note that initiating remote control (by turning on the remote) will
reset any calibration data to defaults and also reset any exception states!
